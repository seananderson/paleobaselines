\documentclass[11pt]{article}
\usepackage{geometry}
\geometry{letterpaper}
%\usepackage[parfill]{parskip}    % Activate to begin paragraphs with an empty line rather than an indent
\usepackage{graphicx}
\geometry{verbose,letterpaper,tmargin=2.54cm,bmargin=2.54cm,lmargin=2.54cm,rmargin=2.54cm}
\textheight 22.0cm
\usepackage{rotating}
%\usepackage[right]{showlabels}
%\renewcommand{\showlabelsetlabel}[1]
%{\begin{turn}{90}\showlabelfont #1\end{turn}}


\title{Supplemental figures for revision 1}
\author{}

\begin{document}
\maketitle

\section{Fossil data sensitivity analyses}

\begin{figure}[htbp]
\begin{center}
\includegraphics[width=5in]{hist-preserv-prob.pdf}
%\includegraphics[width=5in]{hist-single-occ.pdf}
\caption{Frequency distribution of preservation probability by genera in the Neogene fossil record. Our preservation probability calculations included the Neogene stages Lower Miocene, Middle Miocene, Upper Miocene, and Pliocene, Pleistocene, along with the Paleogene and contemporary extinction status. A preservation probability of 1.0, for example, means that a genus was observed in all stages between the first and last stages it was observed in.}
\label{fig:hist-preserv-prob}
\end{center}
\end{figure}

\clearpage

\begin{figure}[htbp]
\begin{center}
\includegraphics[width=\textwidth]{fossil-cull-N.pdf}
\caption{Effect of different minimum preservation probability culls on the distribution of fossil genera with respect to potential extinction risk predictors. Rows 1--2: total number of genera, rows 3--4: number of extinct genera, rows 5--6: proportion of extinct genera.}
%\label{fig:raw-predictor-paleo-culls}
\end{center}
\end{figure}

\clearpage

\begin{figure}[htbp]
\begin{center}
\includegraphics[width=\textwidth]{fossil-cull-N-sing.pdf}
\caption{Effect of culling genera with single (or zero) occurrences in the fossil record in any stage in which we would have expected to observe them on the distribution of fossil genera with respect to potential extinction risk predictors. Rows 1--2: total number of genera, rows 3--4: number of extinct genera, rows 5--6: proportion of extinct genera.}
%\label{fig:raw-predictor-paleo-culls}
\end{center}
\end{figure}

\clearpage

%\begin{figure}[htbp]
%\begin{center}
%\includegraphics[width=\textwidth]{fossil-cull-self-prediction-distributions.pdf}
%\caption{Distributions of predicted risk at precise values of proportion complete (not cumulative thresholds). Same thing in violin and boxplots.}
%\label{default}
%\end{center}
%\end{figure}


\begin{figure}[htbp]
\begin{center}
\includegraphics[width=\textwidth]{fossil-cull-self-prediction-threshold-distributions.pdf}
\caption{Distributions of predicted intrinsic risk for genera in each of the six major marine groups, varying the preservation probability threshold above which genera are included in the Neogene GBM model. A threshold of 0.0, for example, includes all genera, whereas a threshold of 0.5 includes only those genera with preservation probabilities of 0.5 or greater. Plotted are raw GBM risk estimates prior to calibration. Median intrinsic risk declines with more stringent preservation thresholds as rare and/or poorly sampled genera are excluded from the data. The boxes show the interquartile range, the whiskers show 1.5 times the interquartile range, and dots show outliers beyond that range. Lines in the middle indicate the median values.}
\label{fig:ext-boxplots-paleo-culls}
\end{center}
\end{figure}

\clearpage

\begin{figure}[htbp]
\begin{center}
\includegraphics[width=0.9\textwidth]{partial-class-level-median-estimates}
\caption{The same data as Fig.~\ref{fig:ext-boxplots-paleo-culls} but arranged differently. These are medians with 40\% and 60\% quantiles of the `group'-level partial dependence values aggregated by class.}
\label{fig:class-level-partial-order}
\end{center}
\end{figure}

\clearpage

\begin{figure}[htbp]
\begin{center}
\includegraphics[width=0.95\textwidth]{fossil-cull-comparison-continuous.pdf}
\includegraphics[width=0.65\textwidth]{fossil-cull-comparison-group.pdf}
\caption{Partial dependence plots with increasing threshold of preservation probability in the Neogene fossil record. This is a sensitivity test of Fig.~1 in the main text.}
\label{fig:partial-paleo-culls}
\end{center}
\end{figure}

\clearpage

% \begin{figure}[htbp]
% \begin{center}
% \includegraphics[width=0.9\textwidth]{fossil-cull-partial-groups-vs-completeness.pdf}
% \caption{Partial dependence values for different group values plotted against mean proportional completeness. \textbf{not sure how useful this is now}}
% \label{fig:ext-prop-compl-scatter}
% \end{center}
% \end{figure}
%
% \clearpage

\begin{figure}[htbp]
\begin{center}
\includegraphics[width=0.9\textwidth]{pairs-paleo-culls-genera}
\caption{Comparison of genus-level intrinsic extinction risk estimates from GBM models built on different subsets of the Neogene fossil record. \texttt{mean\_observed} refers to the main GBM model used in the paper. \texttt{mean\_observed\_1.0} refers to a model trained on a dataset containing only genera with a 1.0 preservation probability. \texttt{mean\_observed\_sing} refers to a model containing only genera that were observed at least twice in all stages in which we expected to observe them. Each dot represents a prediction for an individual genus in the modern record based on a model trained in the fossil record. Spearman's rank order correlation coefficients are shown in the upper right triangle of panels.}
\label{fig:pairs-genus-paleo-culls}
\end{center}
\end{figure}

\clearpage

\begin{figure}[htbp]
\begin{center}
\includegraphics[width=0.9\textwidth]{pairs-paleo-culls-provinces}
\caption{Comparison of province-level log mean intrinsic extinction risk estimates from GBM models built on different subsets of the Neogene fossil record. Description of this figure is the same as the previous figure except these are biogeographic province-level estimates instead of individual genus-level estimates. These are the data that underly the final maps such as Fig.~3.}
\label{fig:pairs-prov-paleo-culls}
\end{center}
\end{figure}

\clearpage

\begin{figure}[htbp]
\begin{center}
\includegraphics[width=0.6\textwidth]{hotspots-paleo-0-pp}
\includegraphics[width=0.6\textwidth]{hotspots-paleo-1-pp}
\includegraphics[width=0.6\textwidth]{hotspots-paleo-sing-pp}
\caption{Global maps from models built with various culls of the Neogene fossil record. From top to bottom: the base model, only genera with a preservation probability of 1.0, and only genera with at least two observations in all stages for which we would expect their to be observations.}
\label{fig:overall-maps-paleo-culls}
\end{center}
\end{figure}

\begin{figure}[htbp]
\begin{center}
\includegraphics[width=\textwidth]{class-risk-maps-mean-log-ext-paleo-1-pp}
\caption{Global maps of various taxonomic subgroups from a model built on Neogene data containing only genera with a preservation probability of 1.0. Note that the colour scales shift slightly between the main paper Fig.\ 2, this figure, and the next figure.}
\label{fig:class-maps-paleo-cull-1-pp}
\end{center}
\end{figure}

\begin{figure}[htbp]
\begin{center}
\includegraphics[width=\textwidth]{class-risk-maps-mean-log-ext-paleo-sing-pp}
\caption{Global maps of various taxonomic subgroups from a model built on Neogene data containing only genera with at least two observations in all stages in which we would expect to observe them. Note that the colour scales shift slightly between the main paper Fig.\ 2, this figure, and the previous figure.}
\label{fig:class-maps-paleo-sing-pp}
\end{center}
\end{figure}

\clearpage

\section{Contemporary data sensitivity analyses}

\begin{figure}[htbp]
\begin{center}
\includegraphics[width=0.7\textwidth]{obis-samping-vs-predictors}
\caption{Variation among extant genera in intrinsic extinction risk and great circle distance as a function of the number of OBIS records. Overall intrinsic extinction risk declines as the number of OBIS records increases, as expected biologically and under a model in which poorly sampled genera artificially appear to be at elevated extinction risk. Notably taxa with few raw occurrences exhibit a broad range of intrinsic extinction risk values. Overall great circle distance increases with the number of OBIS occurrences; however, above approximately 5 OBIS occurrences genera exhibit the full range of great circle distance values. Consequently we impose a minimum threshold of 5 OBIS occurrences for all analyses presented in the main text and in the SOM investigate the robustness of our results to varying the minimum number of OBIS records.}
\label{fig:obis-sampling-vs-predictors}
\end{center}
\end{figure}

\clearpage

%\begin{figure}[htbp]
%\begin{center}
%\includegraphics[width=\textwidth]{../analysis/obis-sampling-vs-predictors-class-facets}
%\caption{Predictors vs.\ number of OBIS records. Split by class. Remove this?}
%\label{fig:obis-sampling-vs-predictors-by-class}
%\end{center}
%\end{figure}
%
%\clearpage

%\begin{figure}[htbp]
%\begin{center}
%\includegraphics[width=0.7\textwidth]{../analysis/meanext-vs-obis-n-by-prov}
%\caption{Province-level overall mean extinction risk vs.\ number of OBIS records. Line is a loess smoother with 95\% CI. I hadn't circulated this one. Do we want to keep it?}
%\label{fig:ext-vs-obis-n-by-prov}
%\end{center}
%\end{figure}

\clearpage

\begin{figure}[htbp]
\begin{center}
\includegraphics[width=0.9\textwidth]{pairs-obis-culls}
\caption{Comparison of log mean extinction risk for biogeographic provinces in the contemporary record under varying sampling thresholds in the OBIS database. Spearman rank-order correlation values are presented in the upper right cells. Provincial extinction risk estimates in provinces are strongly correlated across all thresholds considered.}
\label{fig:pairs-prov-obis-culls}
\end{center}
\end{figure}

\clearpage

\begin{figure}[htbp]
\begin{center}
\includegraphics[width=\textwidth]{overall-map-obis-culls}
\caption{Fig.~3 from the main paper recreated with varying sampling thresholds in the OBIS database. Note that the colours are rescaled on each map. The pattern has little qualitative change until we reach a threshold around 50 OBIS records.}
\label{fig:overall-maps-obis-culls}
\end{center}
\end{figure}

\begin{figure}[htbp]
\begin{center}
  \includegraphics[width=\textwidth]{class-map-obis-cull-5.pdf}
\caption{Fig.~2 from the main paper recreated excluding all genera with $<$ 5 OBIS records. (TODO note, this will be the new main figure).}
\label{fig:class-map-obis-cull-5}
\end{center}
\end{figure}

\begin{figure}[htbp]
\begin{center}
  \includegraphics[width=\textwidth]{class-map-obis-cull-10.pdf}
\caption{Fig.~2 from the main paper recreated excluding all genera with $<$ 10 OBIS records.}
\label{fig:class-map-obis-cull-10}
\end{center}
\end{figure}

\begin{figure}[htbp]
\begin{center}
  \includegraphics[width=\textwidth]{class-map-obis-cull-20.pdf}
\caption{Fig.~2 from the main paper recreated excluding all genera with $<$ 20 OBIS records.}
\label{fig:class-map-obis-cull-20}
\end{center}
\end{figure}

\begin{figure}[htbp]
\begin{center}
  \includegraphics[width=\textwidth]{class-map-obis-cull-50.pdf}
\caption{Fig.~2 from the main paper recreated excluding all genera with $<$ 50 OBIS records.}
\label{fig:class-map-obis-cull-50}
\end{center}
\end{figure}

\clearpage

\section{Main paper figures (for ease of comparison)}

\begin{figure}[htbp]
\begin{center}
\includegraphics[width=\textwidth]{partial-plot-base.pdf}
\caption{Main paper fig 1}
\label{fig:1}
\end{center}
\end{figure}

\begin{figure}[htbp]
\begin{center}
\includegraphics[width=\textwidth]{class-risk-maps-mean-log-ext.pdf}
\caption{Main paper fig 2}
\label{fig:2}
\end{center}
\end{figure}

\begin{figure}[htbp]
\begin{center}
\includegraphics[width=0.8\textwidth]{hotspots.pdf}
\caption{Main paper fig 3}
\label{fig:3}
\end{center}
\end{figure}

\clearpage
\section{To do}
\begin{enumerate}

\item  Switch labels of map colour scales to multiplicative scale (0.5, 1, 1.5, 2.0 instead of (0.06, 0.12, etc.)
\item Switch main figures to use an OBIS threshold of 5 observations
\item Make the axes in the pairs plots constant across panels

\end{enumerate}

\end{document}