\documentclass[11pt]{article}
\usepackage{geometry}
\geometry{letterpaper}
%\usepackage[parfill]{parskip}    % Activate to begin paragraphs with an empty line rather than an indent
\usepackage{graphicx}
\geometry{verbose,letterpaper,tmargin=2.54cm,bmargin=2.54cm,lmargin=2.54cm,rmargin=2.54cm}
\textheight 22.0cm
\usepackage{rotating}
\usepackage[right]{showlabels}
\renewcommand{\showlabelsetlabel}[1]
{\begin{turn}{90}\showlabelfont #1\end{turn}}


\title{Supplemental figures for revision 1}
\author{}

\begin{document}
\maketitle

\section{Fossil data sensitivity analyses}



\begin{figure}[htbp]
\begin{center}
\includegraphics[width=\textwidth]{fossil-cull-N.pdf}
\caption{Total number of genera (top 2 rows), number of extinct genera (rows 3--4) , and ratio of extinct to total genera for each of the continuous variables at various thresholds of proportion complete in the Neogene fossil record.}
\label{fig:raw-predictor-paleo-culls}
\end{center}
\end{figure}

\clearpage

%\begin{figure}[htbp]
%\begin{center}
%\includegraphics[width=\textwidth]{fossil-cull-self-prediction-distributions.pdf}
%\caption{Distributions of predicted risk at precise values of proportion complete (not cumulative thresholds). Same thing in violin and boxplots.}
%\label{default}
%\end{center}
%\end{figure}


\begin{figure}[htbp]
\begin{center}
\includegraphics[width=\textwidth]{fossil-cull-self-prediction-threshold-distributions.pdf}
\caption{The distribution of the taxonomic partial-dependence component of the GBM models at various levels of proportion complete in the Neogene fossil record. The boxes show the interquartile range, the whiskers show 1.5 times the interquartile range, and dots show outliers beyond that range. Lines in the middle indicate the median values.}
\label{fig:ext-boxplots-paleo-culls}
\end{center}
\end{figure}

\clearpage

\begin{figure}[htbp]
\begin{center}
\includegraphics[width=0.9\textwidth]{partial-class-level-median-estimates}
\caption{The same data as Fig.~\ref{fig:ext-boxplots-paleo-culls} but arranged differently. These are medians with 40\% and 60\% quantiles of the `group'-level partial dependence values aggregated by class.}
\label{fig:class-level-partial-order}
\end{center}
\end{figure}

\clearpage

\begin{figure}[htbp]
\begin{center}
\includegraphics[width=0.95\textwidth]{fossil-cull-comparison-continuous.pdf}
\includegraphics[width=0.65\textwidth]{fossil-cull-comparison-group.pdf}
\caption{Partial dependence plots with increasing threshold of proportion complete in the Neogene fossil record. This is a sensitivity test of Fig.~1 in the main text.}
\label{fig:partial-paleo-culls}
\end{center}
\end{figure}

\clearpage

\begin{figure}[htbp]
\begin{center}
\includegraphics[width=0.9\textwidth]{fossil-cull-partial-groups-vs-completeness.pdf}
\caption{Partial dependence values for different group values plotted against mean proportional completeness.}
\label{fig:ext-prop-compl-scatter}
\end{center}
\end{figure}

\clearpage

\begin{figure}[htbp]
\begin{center}
\includegraphics[width=0.9\textwidth]{pairs-paleo-culls-genera}
\caption{Pairs plots for paleo culls. Each dot is an individual genus.}
\label{fig:pairs-genus-paleo-culls}
\end{center}
\end{figure}

\clearpage

\begin{figure}[htbp]
\begin{center}
\includegraphics[width=0.9\textwidth]{pairs-paleo-culls-provinces}
\caption{Pairs plots for paleo culls by province. The values in the upper-right panels are Spearman's rank-order correlation coefficients.}
\label{fig:pairs-prov-paleo-culls}
\end{center}
\end{figure}

\clearpage


\begin{figure}[htbp]
\begin{center}
\includegraphics[width=0.6\textwidth]{hotspots-paleo-00}
\includegraphics[width=0.6\textwidth]{hotspots-paleo-25}
\includegraphics[width=0.6\textwidth]{hotspots-paleo-50}
\caption{Global maps with various paleo culls. From top to bottom: all, $\ge$ 0.25, $\ge$ 0.50.}
\label{fig:overall-maps-paleo-culls}
\end{center}
\end{figure}

\begin{figure}[htbp]
\begin{center}
\includegraphics[width=\textwidth]{class-risk-maps-mean-log-ext-paleo-25}
\caption{Global class-level maps at a paleo cull of $\ge 0.25$. Note that the colour scales shift slightly between the main paper Fig.\ 2, this figure, and the next figure.}
\label{fig:class-maps-paleo-cull-0.25}
\end{center}
\end{figure}

\begin{figure}[htbp]
\begin{center}
\includegraphics[width=\textwidth]{class-risk-maps-mean-log-ext-paleo-50}
\caption{Global class-level maps at a paleo cull of $\ge 0.50$.}
\label{fig:class-maps-paleo-cull-0.50}
\end{center}
\end{figure}

\clearpage

\section{Contemporary data sensitivity analyses}

\begin{figure}[htbp]
\begin{center}
\includegraphics[width=\textwidth]{../analysis/obis-samping-vs-predictors}
\caption{Predictors vs.\ number of OBIS records. Remove all but top two panels?}
\label{fig:obis-sampling-vs-predictors}
\end{center}
\end{figure}

\clearpage

%\begin{figure}[htbp]
%\begin{center}
%\includegraphics[width=\textwidth]{../analysis/obis-sampling-vs-predictors-class-facets}
%\caption{Predictors vs.\ number of OBIS records. Split by class. Remove this?}
%\label{fig:obis-sampling-vs-predictors-by-class}
%\end{center}
%\end{figure}
%
%\clearpage

%\begin{figure}[htbp]
%\begin{center}
%\includegraphics[width=0.7\textwidth]{../analysis/meanext-vs-obis-n-by-prov}
%\caption{Province-level overall mean extinction risk vs.\ number of OBIS records. Line is a loess smoother with 95\% CI. I hadn't circulated this one. Do we want to keep it?}
%\label{fig:ext-vs-obis-n-by-prov}
%\end{center}
%\end{figure}

\clearpage

\begin{figure}[htbp]
\begin{center}
\includegraphics[width=0.9\textwidth]{pairs-obis-culls}
\caption{Pairs plots for OBIS culls by province.}
\label{fig:pairs-prov-obis-culls}
\end{center}
\end{figure}

\clearpage

\begin{figure}[htbp]
\begin{center}
\includegraphics[width=\textwidth]{overall-map-obis-culls}
\caption{Global maps with various OBIS culls. Note that the colours are rescaled on each map --- I will fix this. Should we also have class-level maps? Did I already make those?}
\label{fig:overall-maps-obis-culls}
\end{center}
\end{figure}

\end{document}
